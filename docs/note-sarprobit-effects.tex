\documentclass[a4paper]{article}
\usepackage[margin=2cm]{geometry}
\usepackage{amsmath,amssymb}
\usepackage[english]{babel}

\title{Computing the marginal effects of the SAR probit and SAR Tobit model}
\author{Stefan Wilhelm}

% latin vector symbols
\def\bfa{\mbox{$\boldsymbol{\mathbf{a}}$}}
\def\bfb{\mbox{$\boldsymbol{\mathbf{b}}$}}
\def\bfc{\mbox{$\boldsymbol{\mathbf{c}}$}}
\def\bfu{\mbox{$\boldsymbol{\mathbf{u}}$}}
\def\bfx{\mbox{$\boldsymbol{\mathbf{x}}$}}
\def\bfy{\mbox{$\boldsymbol{\mathbf{y}}$}}
\def\bfz{\mbox{$\boldsymbol{\mathbf{z}}$}}

% greek vector symbols
\def\bfbeta{\mbox{$\boldsymbol{\beta}$}}
\def\bfepsilon{\mbox{$\boldsymbol{\epsilon}$}}
\def\bfmu{\mbox{$\boldsymbol{\mu}$}}
\def\bftheta{\mbox{$\boldsymbol{\theta}$}}

% matrix symbols
\def\bfZero{\mbox{$\boldsymbol{\mathbf{0}}$}}
\def\bfA{\mbox{$\boldsymbol{\mathbf{A}}$}}
\def\bfD{\mbox{$\boldsymbol{\mathbf{D}}$}}
\def\bfH{\mbox{$\boldsymbol{\mathbf{H}}$}}
\def\bfI{\mbox{$\boldsymbol{\mathbf{I}}$}}
\def\bfL{\mbox{$\boldsymbol{\mathbf{L}}$}}
\def\bfS{\mbox{$\boldsymbol{\mathbf{S}}$}}
\def\bfT{\mbox{$\boldsymbol{\mathbf{T}}$}}
\def\bfW{\mbox{$\boldsymbol{\mathbf{W}}$}}
\def\bfX{\mbox{$\boldsymbol{\mathbf{X}}$}}

\def\bfSigma{\mbox{$\boldsymbol{\Sigma}$}}


\begin{document}

\maketitle

\section{The problem}
In this note we show how to compute the marginal effects (especially the total impacts $M_r(T)$)
for the SAR model and its probit variant \textit{without} inverting the Matrix $\bfS=(\bfI_n - \rho \bfW)$.
We only use the $QR$ decomposition of $\bfS$.\\
The problem with inverting $\bfS$ is twofold. 
First, while $\bfS$ is often a sparse matrix, $\bfS^{-1}$ is not. $\bfS^{-1}=(\bfI_n - \rho \bfW)^{-1}$ will be a dense
$n \times n$ matrix. For a large $n$ this may be problematic both in terms of memory consumption and speed.
Second, matrix inversion is expensive $O(n^3)$ and we already have a $QR$ decomposition of $\bfS$ available.
When computing a summary measure like $\bfS^{-1} 1_n=\bfx$, we can well solve the system $\bfS \bfx = 1_n$ using the QR-decomposition
of $\bfS$.  

\section{Marginal effects in the SAR model}

Suppose we have the spatial autoregressive model (SAR model, spatial lag model) 
\begin{equation}
  \bfz = \rho \bfW \bfz + \bfX \bfbeta + \epsilon, \quad \epsilon \sim N\left(0, \sigma^2_\epsilon \bfI_n\right)
  \label{eqn:sarprobit}
\end{equation}
for $\bfz = (z_1,\ldots,z_n)'$ with some fixed matrix of covariates $\bfX$ $(n \times k)$ 
associated with the parameter vector $\bfbeta$ $(k \times 1)$. 
The matrix $\bfW$ $(n \times n)$ is called the spatial weight matrix and 
captures the dependence structure between neighboring observations
such as friends or nearby locations. 
The term $\bfW \bfz$ is a linear combination of neighboring observations.
The scalar $\rho$ is the dependence parameter and will assumed $\text{abs}{(\rho)} < 1$.
The $k+1$ model parameters to be estimated are the parameter vector $\bfbeta$ and the scalar $\rho$.

Changes of the level of a single variable $x_{ir}$ will impact both the outcome of the same observation $z_i$ and possibly all other
observations $z_j$ ($j \ne i$). So the matrix of possible changes to the $r$-th variables will be
\begin{eqnarray}
E(\partial y / \partial x'_r) & = & (\bfI_n - \rho \bfW)^{-1} I_n \beta_r \\
                              & = & S_r(\bfW) \nonumber
\end{eqnarray}
which is a $n \times n$ matrix.

See \cite{LeSage2009}, section 5.6.2., p.149/150 for spatial effects estimation in MCMC. They propose three scalar summary
measures of the spatial spillovers:
\begin{itemize}
\item average direct impacts: $M_r(D) = n^{-1} tr(S_r(\bfW))$      \\
Efficient approaches for computing the trace of $S_r(\bfW)$ are available, see \cite{LeSage2009} chapter 4, pp.114/115.
\item average total impacts: $M_r(T) = n^{-1} 1'_n S_r(\bfW) 1_n$  \\
 The problem with the total effects is that the matrix $S_r(\bfW)$ is a dense $n \times n$ matrix! 
 For large $n$ this will run into storage problems.
\item average indirect impacts: $M_r(I) = M_r(T) - M_r(D)$\\
 The indirect impacts are directly linked to the total effects, so they may also suffer from the same problems.
 Put it differently, once total effects can be efficiently computed, so can indirect effects.
\end{itemize}

\section{Marginal effects in SAR probit}

See \cite{LeSage2009}, section 10.1.6, p.\,293 for marginal effects in SAR probit.
For the change in $\beta_r$ we have the $n \times n$ effects matrix
\begin{eqnarray}
 \partial E[y | x_r] / \partial x'_r & = & \phi(\bfS^{-1} \bfI_n \bar{x}_r \beta_r) \odot \bfS^{-1} \bfI_n \beta_r \\
                                     & = & \phi(\mu) \odot \bfS^{-1} I_n \beta_r \nonumber \\
                                     & = & \bfD \cdot \bfS^{-1} \cdot \beta_r \nonumber \\
                                     & = & S_r(\bfW) \nonumber  
\end{eqnarray}
where $\bfD$ is the diagonal matrix containing $\phi(\mu)$.\\
\par
We are already solving the equation 
\begin{equation}
  (\bfI_n - \rho \bfW) \mu = \bfS \mu = \bfX \bfbeta \nonumber
\end{equation}   
for $\mu$ using the QR-decomposition of $S$
(as opposed to inverting $\bfS$ as in $\mu = \bfS^{-1} \bfX \bfbeta = (\bfI_n -  \rho \bfW)^{-1} \bfX \bfbeta$),
so we have the QR decomposition of $\bfS$ ready for the calculation of the total effects estimates.

We rewrite the average total effects as
\begin{eqnarray}
\text{average total effects} & = & n^{-1} 1_n' (\bfD \bfS^{-1} \cdot \beta_r) 1_n \\
                             & = & n^{-1} 1_n' (\bfD \bfx) \cdot \beta_r \nonumber \\
                             & = & \text{mean}(\bfD \bfx) \cdot \beta_r \nonumber
\end{eqnarray}
where $\bfx$ is the solution of $\bfS \bfx = 1_n$, obtained from the QR-decompositon
of $\bfS$. The average total effects is then the mean of $(\bfD \bfx) \cdot \beta_r$, which can be furthermore done for all $\beta_r$ in one operation.

\section{Marginal effects in the SAR Tobit model}

\subsection{Marginal effects in the non-spatial Tobit model}

The non-spatial Tobit model can be written as
\begin{eqnarray}
  y^{*} & = & \bfx \bfbeta + \epsilon, \quad \epsilon | \bfx \sim N(0, \sigma^2) \\
	y     & = & \max(0, y^{*})\\
	      & = & \left\{ 
	              \begin{array}{cc}
								  y^{*} & \text{ if } y^{*} \ge 0  \\
									0 & \text{ if } y^{*} < 0
								\end{array}
	            \right.
%\label{eq:standard-tobit}
\end{eqnarray}

According to \cite{Wooldridge2010}, eqn (17.16), p.\,674 the marginal effect is
\begin{eqnarray}
 \partial E[y | \bfx ] / \partial x'_r & = & \Phi(\bfx \bfbeta / \sigma) \beta_r
\end{eqnarray}

\subsection{Marginal effects in the spatial Tobit model}

In \cite{LeSage2009}, section 10.3, the spatial Tobit model is described as
\begin{eqnarray}
  y^{*} & = & (\bfI_n - \rho \bfW)^{-1} \bfX \bfbeta + (\bfI_n - \rho \bfW)^{-1} \epsilon, \quad \epsilon | \bfx \sim N(0, \sigma^2) \\
	y     & = & \max(0, y^{*})\\
	      & = & \left\{ 
	              \begin{array}{cc}
								  y^{*} & \text{ if } y^{*} \ge 0  \\
									0 & \text{ if } y^{*} < 0
								\end{array}
	            \right. \\
	\bfS     & = & \bfI_n - \rho \bfW 
\end{eqnarray}

Proposition:
\begin{eqnarray}
\partial E[y | x_r] / \partial x'_r & = & \Phi(\bfS^{-1} \bfI_n \bar{x}_r \beta_r / \sigma) \odot \bfS^{-1} \bfI_n \beta_r
\end{eqnarray}

\bibliographystyle{abbrvnat}
\bibliography{spatialprobit}

\end{document}

